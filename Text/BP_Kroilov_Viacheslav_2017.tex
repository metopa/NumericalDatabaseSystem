\documentclass[thesis=B,english,hidelinks]{FITthesis}[2012/10/20]

\usepackage[utf8]{inputenc} % LaTeX source encoded as UTF-8
\usepackage{graphicx} % graphics files inclusion
\usepackage{dirtree} % directory tree visualisation
\usepackage[]{acronym} % acronims
\usepackage{float} % floats
\usepackage{indentfirst} % indent in first paragraph
\raggedbottom
\usepackage{amsmath}
\usepackage[all]{hypcap} % for going to the top of an image when a figure reference is clicked
\usepackage[numbers]{natbib}
\usepackage{listings} % for the source code
\usepackage[outputdir=./pdf]{minted} % for the source code
\usepackage{url}
\usepackage{enumitem}
\usepackage{xspace}
\usepackage{cleveref}
\usepackage{tikz}
\usepackage{pgfplots}
\usepackage{booktabs}

\newcommand{\libname}[1]{\textsc{\mbox#1}\xspace}
\newcommand{\numdbname}{\libname{numdb}}
\newcommand{\cndcname}{\libname{cndc}}


\graphicspath{{img/}} % images folder


% % list of acronyms
% \usepackage[acronym,nonumberlist,toc,numberedsection=autolabel]{glossaries}
% \iflanguage{czech}{\renewcommand*{\acronymname}{Seznam pou{\v z}it{\' y}ch zkratek}}{}
% \makeglossaries

\department{Department of Theoretical Computer Science}
\title{Concurrent Memoization System for Large-Scale Scientific Applications}
\authorGN{Viacheslav} %author's given name/names
\authorFN{Kroilov} %author's surname
\author{Viacheslav} %author's name without academic degrees
\authorWithDegrees{Viacheslav Kroilov} %author's name with academic degrees
\supervisor{doc. Ing. Ivan {\v S}ime{\v c}ek, Ph.D.}
\acknowledgements{THANKS (remove entirely in case you do not with to thank anyone)}
\abstractEN{Numerical databases speed up computations by memoizing pairs of an argument and the result, computed by a function with the argument. The canonical numerical database is based on the weighted search tree~-- a combination of the AVL tree and the binary heap. The application of alternative data structures, namely the hash table and the splay tree, is discussed in this thesis. In addition, a new data structure~-- \cndcname{}~-- is introduced. It is similar to the weighted search tree, but all operations are declared as thread-safe.

            Data structures, mentioned above, are implemented in the C++ programming language as a programming library, called \numdbname. The performance of each data structure is measured and the results are compared and discussed.
}
\abstractCS{Numerická databáze zrychluje výpočet ukládáním mezivýsledků do pamětí.
            Kanonická implementace numerické databáze je založená na ohodnoceném
            binárním stromu~-- kombinace AVL-stromu a binární haldy.
            V této práci je diskutována i možnost využití jiných datových struktur,
            jak je Splay-strom a hašovácí tabulka.
            Navíc je zavedená zcela nová datová struktura~-- \cndcname. Podporuje stejné
            operace jako ohodnocený binární strom,
            ale je přizpůsobená k použití ve vícevláknovém prostředí.

            Všechny zmíněné datové struktury jsou implementovány v
            programovacím jazyce C++ v podobě programovací knihovny \numdbname.
            Na závěr jsou uvedené výsledky měření výkonnosti implementovaných
            datových struktur.
}
\placeForDeclarationOfAuthenticity{Prague}
\keywordsCS{numerická databáze, vypočetní optimizace,
            splay strom, hašovací tabulka, datové struktury pro paralelní
            zpracování,
            vícevláknová synchronizace, fine-grained locking, C++}
\keywordsEN{numerical database, computational optimization, splay tree, hash table, concurrent lookup data structure, fine-grained locking, C++}
\declarationOfAuthenticityOption{1} %select as appropriate, according to the desired license (integer 1-6)
% \website{http://site.example/thesis} %optional thesis URL

\setcounter{secnumdepth}{3}

\begin{document}


%%%%%%%%%%%%%%%%%%%%%%%%%%%%% Custom commands %%%%%%%%%%%%%%%%%%%%%%%%%%%%%

\newcommand{\smplimage}[3][1]{
    \centerline{\includegraphics[width=#1\textwidth]{#2.#3}}
}

% \image[size]{diagram and lable name}{extention}{caption}
% \image[1.3]{component_diagram}{pdf}{Component diagram}
\newcommand{\image}[4][1]{
\begin{figure}[H]
    \smplimage[#1]{#2}{#3}
	\caption{#4}
    \label{fig:#2}
\end{figure}
}

\newcommand{\func}[2]{\ensuremath{\textbf{\textit{#1}}(#2)}}
\newcommand{\nsnum}[2]{\ensuremath{\texttt{#1}_{#2}}}
\newcommand{\findop}{\func{find}{}\xspace}
\newcommand{\insertop}{\func{insert}{}\xspace}
\newcommand{\removeop}{\func{remove}{}\xspace}

\newcommand{\classname}[1]{\texttt{#1}\xspace}
\newcommand{\pathname}[1]{\path{#1}}

\mathchardef\mhyphen="2D

%\pgfplotsset{width=6.5cm}

% ==========================================================
% Adapted from TeX.StackExchange.com user Juan A. Navarro's
% solution for itemize as seen at
% https://tex.stackexchange.com/a/4493/327
% ----------------------------------------------------------
\newcommand\blockitem[1][]{%
  \closepage\item[#1]\minipage[t]{\linewidth}%
  \let\closepage\endminipage%
  }
% ----------------------------------------------------------
\newenvironment{block-description}{%
  \description
  \let\olditem\item
  \let\closepage\relax
  %\def\item[1][]{\blockitem}[1]
}{%
  \closepage
  \enddescription
}

\setlength{\parindent}{0em}
\setlength{\parskip}{1em}
\usemintedstyle{vs}
%%%%%%%%%%%%%%%%%%%%%%%%%%%%%%%%%%%%%%%%%%%%%%%%%%%%%%%%%%%%%%%%%%%%%%%%%%%%

\listoflistings

\setsecnumdepth{part}
    \chapter{Introduction}
    
\section{Memoization system}
There is a strong trend nowadays in scientific computations~-- the size of data to be processed increases faster than available computing resources. Different optimization techniques are used to overcome this problem. For example, a memoization technique, which is as follows~-- a result~$R$ that has been computed once by a function~$F$ with the argument~$A$ is stored in memory (together with $A$). Then on consequent calls of $F$ with~$A$, $R$ is retrieved from memory, hence the actual call to~$F$ is omitted. This optimization turns out to be particularly useful when a result of~$F$ takes a long time to be computed (relatively to a time needed to retrieve $R$ from memory), and $F$ is usually called only with a small subset of arguments. Sometimes this technique is also referred to as a caching.

A target audience for this kind of memoization systems is scientific application developers.

The concept of a numerical database, introduced by Park, Draayer, and Zheng\cite{park90}, describes one of such memoization systems. Its main purpose is to ``store and retrieve valuable intermediate information so costly redundant calculations can be avoided.'' Authors also presented a possible implementation, that is based on a weighted search tree.

The main goal of this thesis is to explore different look-up data structures, especially those supporting a concurrent access, which can be used in place of the weighted search tree and compare their performance. Then, the most promising data structures will be packed in a programming library, written in C++ programming language.


\section{Thesis structure}
The remainder of this thesis is structured as follows:
 \begin{description}
 \item[In \cref{ch:pre}] some basic programming concepts and data structures are described. The rest of the thesis relies on these terms.
 \item[In \cref{ch:numdb}] the contributions of~\cite{park90} are recapitulated~-- the numerical database concept and the weighted search tree are explained in~detail.
 \item[In \cref{ch:alt}] all of the main contributions of this thesis are presented. \\ Specifically, the overview of different eviction policies, other than what have been proposed in~\cite{park90}, is provided. Furthermore, alternative data structures, which can be used as a base for the numerical database, are discussed.

 \item [\Cref{ch:impl}] contains general information as well as a more in-depth overview of the \numdbname{} program library. This library provides an implementation of several containers (including concurrent containers), discussed in chapters 2 and 3.

 \item [In \cref{ch:bench}] the results of the performance comparison of the containers are presented and concluded.

 \item [In conclusion] the outcomes of this thesis are recapitulated.

\end{description}



\setsecnumdepth{all}
    \chapter{Preliminaries}
    \label{ch:pre}
    %%%%%%%%%%%%%%%%%%%%%%%%%%%%% Foreword %%%%%%%%%%%%%%%%%%%%%%%%%%%%%

%%%%%%%%%%%%%%%%%%%%%%%%%%%%% Functional requirements %%%%%%%%%%%%%%%%%%%%%%%%%%%%%
\section{Binary Search Tree}

A binary search tree is a data structure that implements \findop, \insertop, and \removeop operations on a set of keys. Key~$K$ can be of any type, that has a total order. Throughout the thesis, trees with distinct keys are discussed, although of course, it is not the necessary condition.

A binary tree consists of nodes. A node~$N$ consists of a key (defined as~\func{key}{N}) and two references to other nodes~-- left child (defined as~\func{left}{N}) and right child (defined as~\func{right}{N}). A reference may contain a link to an existing node or a special value $nil$, that means that the reference is empty. A node that contains a link to a child is called a \emph{parent} of this child node. Nodes may contain other attributes as well, but those are not substantial for this explanation. A node that has no children (\(\func{left}{N} = nil \land \func{right}{N} = nil\)) is called a \emph{leaf}.

From the perspective of the graph theory, the binary tree is a simple oriented acyclic graph, where vertices are represented as nodes and edges are represented as links between a node and its left and right children. Every vertex in such graph has at most one incoming edge, i.e. every node can have at most one parent. Moreover, only one node has no parent~-- this node is called the \emph{root} of a binary tree. The length of the longest path from any leaf to the root is known as the \emph{height} of a binary tree. The \emph{subtree} with a root in $N$ defined as $N$ and a set of nodes that can be reached from $N$ by child links.

Binary \emph{search} tree ($BST$) is a binary tree that satisfies the following condition: for each $N$, subtrees with the root in \func{left}{N} and with the root in \func{right}{N} contain only nodes with keys, that are less or equal than \func{key}{N} and larger or equal than \func{key}{N}, respectively. Using this property, it is possible to implement a fast lookup of a key $K$ in a binary search tree (see \cref{alg:bst_find}).

\newfloat{algorithm}{tbp}{lop}
\begin{algorithm}
\caption{Lookup in $BST$}\label{alg:bst_find}
\begin{algorithmic}[1]
  \Procedure{Find}{$root,K$}\Comment{The node with key = $K$ or $nil$}
    \State $node\gets root$
    \While{$node\not= nil$}
    \If{$K = \func{key}{node}$}
      \State \textbf{return} $node$
    \ElsIf{$K < \func{key}{node}$}
      \State $node \gets \func{left}{node}$
    \Else \Comment{$K > \func{key}{node}$}
      \State $node \gets \func{right}{node}$
    \EndIf
    \EndWhile
    \State \textbf{return} $nil$\Comment{The node was not found}
  \EndProcedure
\end{algorithmic}
\end{algorithm}

It is easy to prove, that the complexity of this search algorithm is~$\mathcal{O}(\func{height}{T})$, assuming that key comparison takes~$\mathcal{O}(1)$. Furthermore, two remaining operations of a binary tree, \insertop and \removeop, are implemented in the same fashion, with partial changes, and both of those operations have complexity~$\mathcal{O}(\func{height}{T})$. Their implementation is described in detail in~\cite{sedgewick}.

It is clear that the performance of all $BST$ operations  directly depends on the tree height, that can vary between $\mathcal{O}(\func{size}{T})$, where~$\func{size}{T}$ is the count of nodes in~$T$,
%(the case when every node, except the last, has exactly one child; a $BST$ has the same structure as a linked list)
 and $\mathcal{O}(\log(\func{size}{T}))$ in case of a complete binary tree~\cite{complete_bt}. Therefore, a BST will maintain optimal operation time only if its structure is close to a complete binary tree and the height is close to a $c \times \log(\func{size}{T})$, where $c$ is a constant factor greater or equal~$1$. To keep the height logarithmic, even in a worst-case scenario, the \emph{tree rebalancing} has been invented. The idea is that a tree keeps track of its structure and if it is not optimal, then the rebalancing is applied to restore an optimal structure. The rebalancing can be achieved through the \emph{tree rotation}\cite{sedgewick}~-- the operation, that swaps a node with its parent in a way, that preserves the $BST$ property.


\section{AVL Tree}

AVL tree was invented in 1962 by Georgy Adelson-Velsky and Evgenii Landis\cite{avl_tree}. It is a classic example of a self-balancing $BST$. In fact, the height of the AVL tree is never greater than $1.4405\times \log(\func{size}{T}) - 0.3277$ \cite[p.~460]{knuth3}.

Self-balancing is achieved with the following approach: every node holds a difference between the heights of its left and right subtrees; this difference is called \emph{balance factor}.
AVL property stands that every node has balance factor $-1$, $0$ or~$1$. After every operation, that modifies the tree structure~-- \insertop and \removeop, balance factors are updated.
If at any step the balance factor happens to be $-2$ or $2$, a rotation or a double rotation is applied.
The rotation adjusts the heights of the left and right subtrees and, consequently, restores the AVL property. The exact AVL tree implementation is described in \cite{sedgewick} and \cite{cormen}.

\section{Splay Tree}
\label{sec:splay}
Another approach on tree balancing is presented in the Sleator and Tarjan work\cite{splay_tree} - ``The efficiency of splay trees comes not from an explicit structural constraint, as with balanced trees, but from applying a simple restructuring heuristic, called splaying, whenever the tree is accessed.'' By term \emph{splaying} they mean the process of using rotations (similar to ones in the AVL tree) to bring the last accessed node to the root. Sleator and Tarjan proved that by using this technique, all three basic operations (\findop, \insertop and \removeop) have a logarithmic time bound. Another benefit of splaying is that the most frequently accessed items tend to gather near the root, therefore improving access speed, especially on skewed input sequences~-- the sequences, in which only a small number of items are accessed often while other items occur rarer.

Even though splay trees show several interesting theoretical properties, in practice they are outperformed by more conventional BSTs, like AVL and Red-Black tree\cite{splay_overview}. This is due to the fact that in the splay tree the structure of the tree is altered on every operation, including find operation, while AVL, for instance, modifies the tree only during insertions and removals. A typical use scenario for those data structures is a scenario, where a vast majority of operations is search operations, while updates are not so often. AVL and Red-Black trees happen to be faster because they execute fewer instructions per find operation. Moreover, they do not make any writes to memory during the lookup, and, as a consequence, there is lower load on the memory bus and system cache. Further researches on splay trees were focused in the main on how to reduce the number of rotations during splaying. An extensive overview of those optimizations is provided in \cite{splay_overview}. One of the described techniques, the partial splaying is a modification of a conventional splay tree, where every node contains a counter that denotes a total count of accesses to this node. As usual, splaying is performed on every access, but the node is splayed only until its access count exceeds an access count of its parent. W. Klostermeyer showed that this modification does not gain any noticeable advantage over a standard splay tree \cite{partial_splaying}. However, partial splaying and other derived modifications can have some interesting properties specifically in application to a numerical database. It will be discussed in \cref{ch:alt}.

\section{Hash Table}

Hash table is another popular data structure that implements dictionary abstract data type. It uses an entirely different approach on item storage and lookup. A hash table allocates a contiguous array $A$, which size is bounded by the expected number of items to be stored, often multiplied by the \mbox{\emph{load~factor}~$\alpha$}.

Firstly, let’s look at a simplified case: the key $K$ that is used in a hash table is of an integer type. Having $A$, $K$ and the value $V$, associated with $K$, it is possible to use a remainder of the division of $K$ by the size of $A$ as an index in $A$. Then, $V$ will be stored in $A$ at this index. This approach would give the best performance possible, as the $V$ can be retrieved immediately and \emph{the~lookup time does not depend on~the~total count of~items} in the hash table. However, since the $modulo$ operation has been used, there can be several keys that point at the same index in $A$. This circumstance is called \emph{collision}.

To deal with a collision, it is necessary to store $K$ itself together with $V$, so that in case of a collision it would be possible to tell if the stored $V$ is actually associated with the $K$ or another $K'$, that collides with $K$. Secondly, one must pick a strategy on how to deal with the case when two different keys, $K_1$ and $K_2$, that point at the same index are inserted. There are two ways to deal with collisions:
\begin{description}
\item[Separate chaining (open hashing)]-- each element in $A$ is a linked list (or another data structure), that stores all pairs $\langle K, V\rangle$ that collides.
\item[Linear probing (closed hashing)]-- if during insertion of $K$ in $A$ at the index $i$ a collision occurs ($i$ is already occupied), a special function $F$ is used to determine the second index at which $K$ can be inserted. If it is also occupied, 3rd and all consequent positions, generated by $F$, are used to try to insert the element.
\end{description}

The approach described above can be generalized on keys of any type $T$. It is achieved with the help of a \emph{hash function}. This function takes an argument of type $T$ and maps it to an integer, called \emph{hash}. This function must satisfy two properties:

\begin{block-description}
\blockitem[Determinism]-- it should \emph{always} map the same input to the same hash.
\blockitem[Uniformity]-- if used with a uniformly generated random sequence of objects on input, the hash function should produce a uniform sequence of hashes.
\end{block-description}

In \cite{sedgewick} Sedgewick and Wayne provide the detailed explanation of collision avoidance strategies as well as general information about hash tables. More information about hash function properties and hash function construction is presented in \cite{knott}.

\section{Coarse-grained Locking}

A trivial way to parallelize a sequential data structure is to eliminate a concurrent access at all. It can be achieved with a single mutual exclusion lock~-- \emph{mutex}.
While a thread holds a mutex, no other threads can lock the same mutex.

The sequential data structure is wrapped into the helper type, that locks the mutex in the beginning of every operation and releases it in the end, so that only one thread can access the data structure at a time, no matter how many threads are involved.
This approach is called the \emph{coarse-grained locking}, in contrast with the \emph{fine-grained locking}, where many locks are used and each lock protects only a part of the data structure, so other threads can freely access other parts.

Pros of this approach is a very trivial implementation and the absence of any special requirements on the underlying data structure. However, coarse-grained locking is only suitable when a data structure has a support role in the program and is used occasionally. If the data structure is the key element of the application, then a single lock becomes the bottleneck in the software, drastically decreasing program scalability. In this case one should use more sophisticated parallelization approaches.

\section{Binning}
\label{sec:pre_bin}
The evolution of coarse-grained locking is the \emph{binning}. The main drawback of the previous approach is that a single lock becomes the main point of contention between threads. One way to cope with this is to increase the number of locks. In contrast with the fine-grained locking, the binning does not involve any modifications of the underlying container.

Firstly, the numbers of bins~-- independent data structure instances~-- is chosen. Then a mapping between the item domain and a bin number is introduced. The mapping should yield a uniform distribution of mapped values. Every item is stored only in its assigned bin. Every bin has its own mutex, therefore the access to every bin is serialized. But since items are mapped uniformly, it is expected to produce much less contention than in case of a single lock.

\section{Fine-grained Locking}

The fine-grained locking usually offers better scalability, than the previously discussed approaches. Instead of a single lock, many mutexes are used simultaneously. Every mutex protects its part of data. The contention between threads is lower as it is unlikely several threads will access the same portion of data at the same time. However, this is true only if every portion has the same probability of being accessed (like in a concurrent hash table). In some data structures, typically binary trees, there are some nodes that are accessed (and are locked before access) oftener than others, e.g. a root in a binary tree. \emph{Substantial modifications} got to be made to the data structure to integrate the fine-grained locking. Sometimes the overhead added by this approach is so big, that it brings to naught any potential speed-up. Fine-grained locking is not a silver bullet, but usually it offers a reasonable trade-off between implementation complexity and application scalability.


    \chapter{Numerical database}
    \label{ch:numdb}
    In this chapter the contributions of~\cite{park90}~-- the numerical database system and the wighted search tree~-- are explained in~detail.

\section{Concept}
In the year 1990 S.C.~Park, J.P.~Draayer, and S.-Q.~Zheng introduced a memoization system, called \emph{numerical database}\cite{park90}. Like every memoization system, its primary goal is to reduce costly redundant calculation. The main idea behind this concept is to design a data structure that stores a limited number of items~-- key-value pairs~-- and provide an efficient way to retrieve, insert and remove items. A value is associated with the key that can be used to calculate the value by a function. The complete process of the item retrieval is demonstrated in \Cref{alg:numdb_get}.

%\newfloat{algorithm}{tp}{lop}
\begin{algorithm}[t]
\caption{Numerical database item retrieval}\label{alg:numdb_get}
\begin{algorithmic}[1]
  \Procedure{Retrieve}{$numdb, K$}
    \State $(V, found) \gets numdb.\func{find}{K} $
    \If {$found = true$}
      \State $numdb.\func{update\_priority}{K} $ \Comment{Can be embedded into \findop}
      \State \textbf{return} $V$
    \EndIf
    \State \Comment{The item was not found and must be recalculated}
    \State $(V, priority) \gets numdb.\func{call\_user\_function}{K}$

    \If {$numdb.\func{is\_full}{}$}
      \State $numdb.\func{evict\_item}{}$ \Comment{Remove the item with the lowest priority}
    \EndIf
    \State $numdb.\func{insert}{K, V, priority}$
    \State \textbf{return} $V$
  \EndProcedure
\end{algorithmic}
\end{algorithm}

\section{Priority}
\label{sec:wst_priority}

Each item has its assigned priority. When accessed, the priority of the node is updated. Park et. al. define priority as follows~-- ``A good priority strategy should enable the frequency of a data item and its intrinsic value to be incorporated into its assigned priority.'' Initial priority can be supplied by an external algorithm or computed using some heuristics. For example, the heuristic can be based on the time it took to compute the value: some items can take much more time to be calculated than others, then these elements should be kept in the database even if they are accessed relatively rarely.

When a numerical database reaches its maximum capacity, the item with the lowest priority is removed from the database prior to the next insertion. This implies that the database should support queries on the item with the minimum priority. The way it is achieved in the original proposal is discussed in \Cref{sec:wst}.
\\

Park et. al. introduced a space-optimal representation of the item priority. To distinguish this representation from others, it will be called the \emph{weighted search tree priority} (WST priority). This representation is stored in a single 32 bits long unsigned integer, but combines both the base priority and the hit frequency at the same time. The first 8 bits of the number are reserved for the base priority while remaining 24 bits contains hit frequency, multiplied by a base priority.
Therefore, the actual priority value equals to
\begin{equation}
hit\_frequency \times base\_priority \times 256 + base\_priority
\end{equation}

Another advantage of the WST priority is a simple adjustment procedure when the hit frequency is updated (see \Cref{alg:wst_priority1}). However, this representation also has some drawbacks that will be discussed in \Cref{sssec:improved_wst}.
%\newfloat{algorithm}{tbp}{lop}
\begin{algorithm}[t]
\caption{$WST$ priority update}\label{alg:wst_priority1}
\begin{algorithmic}[1]
  \Procedure{Update}{$priority$}\Comment{4 bytes long unsigned integer}
    \State $base\_priority\gets priority \mathrel{\&} \nsnum{FF}{16}$\Comment{Hexadecimal number}
    \State $priority \gets priority + base\_priority \times \nsnum{100}{16}$
    \State \Comment{Multiplication shifts the number 8 bits to the left}
    \State \textbf{return} $priority$
  \EndProcedure
\end{algorithmic}
\end{algorithm}

\section{Weighted Search Tree}
\label{sec:wst}
Park et. al. also proposed a data structure, called weighted search tree (\emph{WST}), that can be used as a base for the numerical database. The weighted search tree is a combination of two well-known data structures~-- the AVL tree and the binary heap. Each of them fulfills its purpose:
\begin{block-description}
\blockitem [AVL tree] is used for a fast item lookup, insertion, and removal in $\mathcal{O}(\log{\func{size}{T}})$ time.
\blockitem [Binary heap] is used to maintain priorities of nodes. Specifically, it provides an ability to find a node with the lowest priority in $\mathcal{O}(1)$ and can perform insertions and deletions in $\mathcal{O}(\log{\func{size}{T}})$ time.
\end{block-description}

Weighted search tree holds all its nodes in a single contiguous array. These nodes are ordered in the same way as in a regular minimal binary heap:
\begin{equation}
\begin{aligned}
  \forall N \in \textsc{Heap},{}
     & \func{priority}{N} < \func{priority}{\func{heapLeft}{N}} {} \land \\
     & \func{priority}{N} < \func{priority}{\func{heapRight}{N}}
\end{aligned}
\end{equation}

\func{heapLeft}{N} and \func{heapRight}{N} of $N$ with index $i$ are defined as $(2 \times i)$ and $(2 \times i + 1)$ respectively.


The difference is that each node is an AVL tree node at the same time~-- it stores links to its left child, right child, and parent node.

With a given structure the three basic operations are defined as follows:
\begin{block-description}
\blockitem[\findop] is the same as in any binary tree~-- it is begins at the root and then continues as described in \Cref{alg:bst_find}.
\blockitem[\insertop] consists of two steps:
  \begin{enumerate}
  \item The node is inserted into the binary search tree, and the tree is adjusted using AVL rotations. Obviously, these rotations only change pointers inside nodes. Therefore, they do not affect the binary heap structure.
  \item The node is inserted into the binary heap. It is done using the \func{heapify}{} operation\cite[p.~346]{sedgewick}. Note that heap adjustments would \emph{reorder nodes} so that binary search tree pointers would point to wrong nodes and the whole \emph{tree would become ill-formed}. To avoid this, special care should be taken when swapping nodes~-- one should also check and adjust tree pointers during swaps.
  \end{enumerate}
\blockitem[\removeop] is similar to the insertion, but steps are performed in the reverse order~-- at first, the element is removed from the heap, then from the tree. Again, heap adjustments should be performed with respect to the tree structure.
\end{block-description}

\section{Known Implementations}

There are several known implementations of the numerical database concept.

The reference implementation was made by Park, Bahri, Draayer, and Zheng. A complete source code is available at \cite{wstree}. It was implemented in Fortran programming language. However, the source code cannot be compiled with any modern Fortran compiler\cite{masat}. Therefore, this implementation is mentioned here but cannot participate in the benchmark. The \numdbname library contains the implementation of the weighted search tree, recreated basing on \cite{park90} and \cite{park94}.

Another implementation has been developed and described by Miroslav Masat in his bachelor’s thesis \cite{masat}. Sources are available at \cite{ccherish}. However in this particular implementation item priorities are defined solely by the user and are not updated afterward. Therefore it is only applicable in scenarios when the item importance is known beforehand. A canonical numerical database has a broader field of applications.

Both of the implementations are designed for a single-threaded environment only. It is a huge drawback since it is natural to run scientific applications on many-core systems. Therefore, a memoization system must support concurrent access in order to be usable in real-world applications. It was achieved in the \numdbname library.


    \chapter{Numerical database variations}
    \label{ch:alt}
    \section{Priority}

\subsection{Improved WST Priority}
\label{sssec:improved_wst}
The weighted search tree priority representation, discussed in \cref{sec:wst_priority}, has some drawbacks that probably were irrelevant at the time when \cite{park90} was presented. The problem is that keeping the hit counter in only 24 bits (even more, the hit counter multiplied by the base priority) would result in an integer overflow sooner or later. For example, imagine the case when the base priority is the maximum possible~-- 255. Binary representation is \nsnum{000000FF}{16}. After 65793 hits, the priority will have its maximum value~-- \nsnum{FFFFFFFF}{16}. If another hit counter adjustment is made an overflow occurs, producing the value \nsnum{000000FE}{16}. Therefore, the maximum priority becomes very low, and even the base priority has been changed. There are at least two possible solutions:
\begin{description}
\item[Use larger counter]-- store the WST priority in a 64-bit integer. Then an overflow of a 56-bit counter is unlikely, not to say impossible~-- even with the maximal priority, an overflow will occur only after the 282578800148737th insertion. It would take days to make so many adjustments, even if the processor performs only these adjustments and nothing else (which is at least impractical). Nevertheless, a certain disadvantage is that the memory overhead per each node is increased by 4 bytes.
\item[Perform a saturated addition] when adjusting the hit counter (\cref{alg:wst_priority2}). The saturated addition is the addition that yields the expected result if no overflow occurs during the operation and the maximum number for the given operand size otherwise. The drawback of this method is the increased computation time.
\end{description}

The difference between these two methods is the common trade-off between time and space. Since the total count of items that can be stored in a database with the limited memory available is the crucial characteristic of a database, the preference is given to the \emph{saturated addition} method.
\newfloat{algorithm}{tbp}{lop}
\begin{algorithm}
\caption{$WST$ priority update with saturation}\label{alg:wst_priority2}
\begin{algorithmic}[1]
  \Procedure{UpdateSaturated}{$priority$}\Comment{4 bytes long unsigned integer}
    \State $base\_priority\gets priority \mathrel{\&} \nsnum{FF}{16}$
    \State $new\_priority\gets base\_priority$
    \Comment{8 bytes long unsigned integer}
    \State $new\_priority \gets new\_priority \times \nsnum{100}{16}$
    \State\Comment{Maximum possible result is \nsnum{FFFFFFFF00}{16}}
    \If{$new\_priority < \nsnum{FFFFFFFF}{16}$}
    \Comment {Maximum value for 4 bytes}
      \State $priority \gets new\_priority + base\_priority$
    \Else
      \State $priority \gets \nsnum{FFFFFF00}{16} + base\_priority$
    \EndIf
    \State \textbf{return} $priority$
  \EndProcedure
\end{algorithmic}
\end{algorithm}
%%%%%%%%%%%%Simple priority?

\subsection{Priority Aging}
\label{sssec:priority_aging}
The WST priority has one more drawback~-- it is suitable only for static input distributions~-- distributions, which mean remains constant during the execution. However, if the mean is known beforehand it is possible to construct a static optimal BST\cite[p.~442]{knuth3}, that will be more efficient than any dynamic lookup data structure.

A numerical database with the WST priority performs poorly on the time-varying distribution~-- a distribution which mean changes over time~-- while this type of distributions is more common in the real world applications. The problem arises from the fact that the priority does not reflect when an item was accessed for the last time. The worst-case scenario is the following: an item is added to a database, then it is accessed frequently hence its priority rises to the maximum, and then is not used over a long time. During the runtime, several items like this can appear. Even though at some point they are not accessed anymore they are still the most valuable items from the perspective of the database hence they will be kept much longer than other items. This pollutes the database with elements that are stored but not used.

There are several ways to cope with the problem. First of all, it is possible to use an entirely \emph{different eviction strategy}, the one that is not based on the item priority. Some of these strategies are described in the following section. Another solution is to adjust a priority not only when the corresponding node is accessed but also when it is visited (during the lookup of another item). For example, when searching in a binary search tree, the priority of the node $N$ that is being searched is increased while priorities of nodes, lying on the path between the root of the tree and $N$, are degraded.

This mechanism may not be effective with the AVL tree because in the AVL tree the order of the nodes does not correlates with node priorities. However, it seems very promising in application to splay trees~-- by applying this mechanism, the nodes near the root can stay there if only they are constantly accessed.

\section{Alternative Eviction Policies}
The canonical weighted search tree always chooses the node with the minimum priority for the deletion. However, it is only one of many possible eviction strategies. Some other strategies are presented in this section. From general ones, like LRU, to those exploiting the lookup data structure internals to find the least valuable item.

\subsection{LRU Policy}
The Least-Recently-Used policy tracks every access to the items and sorts them by access order. Then it evicts the item that was not accessed for the longest time. The common implementation is based on a doubly-linked list. When an item is accessed its corresponding node in the LRU list is moved into the head of the list. Then the least-recently-used node is the one in the tail of the list. When a new item is added, it is inserted in the head of the LRU list.

\subsection{LFU Policy}
\label{sssec:lfu}
The Least-Recently-Used item policy fulfills the same purpose as the LRU policy. But when it decides which item should be evicted, the access frequency is also taken in account in addition to the last access time (LRU uses only the latter property).

\subsection{Splay Policy}
\label{sssec:spolicy}
A splay tree tends to keep the most frequently accessed items near its root. By relying on this property, it is possible to eliminate a separate data structure that manages item priorities. When an eviction is performed, one of the bottom nodes is chosen for eviction. Even though this strategy may not choose an optimal node every time, it is expected to perform effectively on average. This approach yields the lowest memory overhead per node among all tested data structures.


\section{Alternative Sequential Containers}

\subsection{Hash Table}
One of the data structures that can be used in place of a weighted search tree is the hash table. Hash tables have faster than balanced BSTs lookup time under most workloads. What is more, a node in a hash table has lower memory overhead than a binary tree~-- using open hashing based on a double linked list every node stores only 2 pointers compared to 3 in a binary tree node and using closed hashing implies that no pointers are stored at all.

However closed hashing can not be used because when the hash table is almost full a lot of unsuccessful probes occur before a suitable index is found. Usually, this problem is solved by rehashing~-- if the count of probes exceeds a certain limit, the hash table is expanded. However, it is impossible in the numerical database since the amount of available memory is preset and cannot be exceeded.

On the other hand, limited memory is rather an advantage for the open hashing. If the total amount of memory available is known beforehand, then a hash table with open hashing can be preallocated to its maximum size and be never rehashed after. This, in turn, allows a concurrent version of a hash table to be simplified as the concurrent rehashing is one of the hardest problems to cope with.

\subsection{Splay Tree}
Another data structure that looks promising is the splay tree.
As it was mentioned in \cref{sec:splay}, usually splay trees tend to be slower than AVL. However, in application to the numerical database, it is possible to exploit the fact that the least valuable nodes are usually gathered in leaves of the tree. Therefore it is is possible to eliminate a binary heap from a numerical database and to use the splay policy, as described in \cref{sec:spolicy}. What is more, it is possible to implement a numerical database using a concurrent splay tree\cite{cb_tree}.

\section{Alternative Concurrent Containers}
\label{sec:concurrent_containers}
\subsection{Coarse-grained Lock Adapter}
The \numdbname library provides a universal adapter, that wraps a sequential container and protects it by using a coarse-grained lock. It has the same interface, as a usual numerical database container. All methods follow the same structure:
\begin{enumerate}
\item the mutex is locked
\item the call is forwarded to the underlying container
\item the mutex is released
\end{enumerate}

\subsection{Binning Adapter}
The binning adapter class realises the binning concept (described in \cref{sec:pre_bin}). It is very similar to the coarse-grained lock adapter, however, it encapsulates several instances of a container and several mutexes.

The number of bins is passed in the class constructor. The mapping is defined as  $\func{hash}{K}\bmod \texttt{bin count}$. Every bin is represented by a sequential container, e.g. the weighted search tree. In order to preserve the memory limit, all available memory is equally divided between all bins.

\subsection{CBDC}
For the purposes of the \numdbname library, the original concurrent container, called \emph{Concurrent Numerical Database Container}~-- \cndcname, has been developed. It defines 3 thread-safe operations~-- \findop, \insertop, and \func{removeMin}{}. Thread-safeness is achieved through the fine-grained locking approach. \cndcname is based on concurrent versions of the hash table and the binary heap. Sequential benchmarks (\cref{sec:seqbench}) proved that the combination of a hash table and a binary heap outperforms numerical databases, that are based on the LRU and LFU eviction strategies.

Fine-grained locking hash table implementation is much simpler compared to a similar concurrent BST. A lock is assigned to every hash table bucket (or every $k$ buckets) and every operation inside the bucket requires to lock the corresponding mutex. Since an operation in one bucket never interferes with any other buckets, only one lock is needed per operation, while other threads can operate on other buckets at the same time. Therefore, the overhead added by locking is much smaller, than in BST, where up to $\log{\func{height}{T}}$ mutexes has to be locked on every operations.

There are several known binary heaps with fine-grained locking (\cite{concurrent_heap1}, \cite{champ} and more). The \cndcname is based on the \libname{champ} binary heap, developed by Tamir, Morrison, and Rinetzky \cite{champ}. Unlike the majority of concurrent binary heaps, \libname{champ} allows priorities to be updated after an insertion.

In the following section, two types of locks are distinguished~-- the \emph{bucket} mutex (the one, that protects a single bucket in a hash table) and the \emph{heap} mutex (the one, that protects a single item in a heap. Every hash table node has a link to the corresponding heap node and vice versa. \cndcname operations are defined as follows:


\begin{block-description}
\item[\findop] consists of the following steps. At first, calling thread locks the corresponding bucket mutex. The requested item is searched in the bucket.

If the item is found, its priority is updated.
Before updating the corresponding heap mutex must be locked.
After locking the heap mutex the link to the heap node is checked again. If it has changed, the heap lock is released and the operation is repeated. Double check is required, because another thread can change the link even in case it does not hold the bucket mutex.

However, the heap mutex is required to be locked prior to the link update. Therefore, when a thread holds a heap mutex it is guaranteed, that no other thread can change the link between the heap node and the hash table node.

When the node is locked, the priority is updated and the \func{bubbleDown}{} (as defined in \cite{champ}) is performed. \func{bubbleDown}{} internally releases the heap and bucket locks.

\item[\insertop] has a structure, similar to \findop.
The corresponding bucket mutex is locked.
New item is inserted into the hash table. After that, the item is inserted in the binary heap (at the last index). Before the insertion is performed, the heap lock of the last index is locked.

The bucket mutex is released. It is possible to do this so early, because the \emph{heap} lock will be held till the end of the operation. While it is locked, no other thread can execute any operation on the same node.

Finally, \func{bubbleUp}{}\cite{champ} is performed. It propagates the node down until the heap invariant is restored.

\blockitem[\func{removeMin}{}] evicts the item with the lowest priority. This operation is decomposed into three independent parts.

\begin{enumerate}
\item The item is evicted from the heap.
\item The hash table node is marked as \emph{deleted}. Otherwise, another thread can access the item and start the priority update routine. Since the node does not exist in the heap anymore, the thread will enter into an infinite loop. Marking solves the problem as follows~-- the marked node can still be accessed by other threads, however, they will skip the priority update stage for the node.
\item The item is removed from the hash table.
\end{enumerate}

\end{block-description}
Search operation can be improved by using some technique, that would allow to release the bucket lock earlier, prior to \func{BubbleDown}{}. Such technique has not been developed yet.



    \chapter{Library implementation}
    \label{ch:impl}
    In the following chapter the \numdbname library is described. At first, global design decisions are explained. Then the most important classes are analyzed and the particular tricks and optimizations used in the implementation are described.

\section{Chosen Technologies}

The library is written in the C++ programming language. It has been chosen by the following criteria:
\begin{itemize}
\item C++ is a compiled language. Dynamic and managed languages have a huge runtime overhead, that significantly affects general application performance, while not providing any serious advantage (at least from the perspective of scientific applications).
\item Modern C++ compilers applies lots of advanced code optimizations, increasing the gap between compiled and dynamic languages even further.
\item C++ has an advanced template system that helps to write highly reusable and extendable code, while adding no overhead in runtime.
\end{itemize}
The library uses some language and standard library features that have been introduced in C++14 standard. Therefore, the compiler must be C++14 conformant.

The build process is managed by the \pathname{CMake} software, one of the most popular tool in this category. A great number of other projects also use \pathname{CMake} for managing the build process. It is trivially to embed a one \libname{CMake}-based project into another.

\pathname{CMake} uses \pathname{CMakeLists.txt} file, that contains the project definition, to generate a make file (other generators are also supported). Then, a program or a library can be built by invocating the GNU \pathname{make} utility.

\section{Project Structure Overview}


\subsection{Library}
The library itself is divided into:
\begin{itemize}
\item [\pathname{include/numdb}] folder containing header files
\item [\pathname{lib}] folder with source files (that can be compiled separately and merged with the user program during the linkage stage)
\item [\pathname{test}] folder containing unit tests
\item [\pathname{benchmark}] folder with the benchmark program, that has been used to perform performance evaluation (\cref{ch:bench})
\end{itemize}
Since the library heavily relies on templates, the majority of code is placed in the header files.


\subsection{External Libraries}
This library relies on several support libraries. They are distributed along with the sources (except \libname{Boost.Math}) in the \pathname{3rdparty} folder. Since all of them use \pathname{CMake} software for building process, they are compiled automatically with the main project.
\begin{itemize}[leftmargin=2cm]
\item [\libname{function\_traits}] extends C++ standard library metaprogramming capabilities by defining the type trait that can \emph{deduce argument types} of a provided functor object.
\item [\libname{murmurhash2functor}] library contains \classname{murmurhash2} hash function\cite{murmurhash} implementation and wraps it into the interface, similar to the \classname{std::hash}.

         The demand for the \classname{std::hash} replacement is dictated by the fact that the standard hash is not suitable for the hash table~-- \classname{std::hash} for an integer number is defined as the number itself (at least on some compilers\cite{std_hash}); under certain circumstances, this can lead to a high number of hash collisions.
\item [\libname{Google Benchmark}] framework is used as a benchmark starter. Its responsibility is to run functions that are to be benchmarked, measure their running time and other metrics, and encode the result into one of common data formats (e.g. \classname{JSON}).
\item [\libname{Google Test}] framework enables unit testing in the project. It provides some helper function and macros to simplify writing of unit tests as well as a common facility to run and evaluate tests.
\item [\libname{Boost.Math}] provides some statistical functions that are required by the benchmark program. This is the only library that is not bundled with the project and must be installed separately.
\end{itemize}
\libname{function\_traits} and \libname{murmurhash2functor} are required by the library itself. Other libraries are needed for testing/benchmarking only.

\section{Source Code Overview}
In the following section individual parts of the library are analyzed. The numerical database is called \emph{function cache} in this implementation because it has a cleaner meaning than the less known numerical database term.

\subsection{numdb.h}
This is the main entry point of the \numdbname library. Users need to include this file into their project to gain access to the data structures provided by the library.
\subsection{function\_cache.h}
\classname{FunctionCache} is the main class that realizes the numerical database concept. However, it does not determine how items are stored~-- it takes a container type as a template parameter and stores all items in the instance of the container provided.

This class defines helper function, that are common for all numerical databases, and type definitions, e.g. \classname{args\_tuple\_t}~-- a tuple that can store arguments of a function call. What is more, the item retrieval operation, as described in the \cref{alg:numdb_get}, is implemented with actual calls to the lookup and insertion routines forwarded to the underlying container.

\classname{FunctionCache} is responsible for calling the provided user function in case the requested item was not found in the container. All invocations are timed with a system clock; then the duration is converted into an initial priority that is assigned to the item (initial priority generator is used).

\subsection{initial\_priority\_generator.h}
This header file contains classes that are responsble for computing the initial item priority basing on the duration of the current user function call and durations of previous calls.

\classname{MinMaxPriorityGenerator} calculates a priority as a linear interpolation between the minimum and the maximum values. \classname{RatioPriorityGenerator} calculates it as a proportion to the current average value. The latter scheme proved to be fairer and is used in the final implementation.

What is more, both schemes have an adaption mechanism~-- the average is divided by 2 every $N$ iterations, so that the latest input would have a bigger influence on the final result than data from previous periods. At the same time, the historic data is not discarded completely.

\subsection{fair\_lru.h and fair\_lru.cpp}
\classname{FairLRU} class implements an alternative eviction strategy~-- the item accessed the least recently among all items is always chosen for eviction. It is achieved by maintaining a doubly linked list of all nodes. When a node is inserted, it is placed in the tail (end) of the list. When a node is accessed, it is extracted from its current position, then inserted in the tail of the list. Therefore, the least recently used node appears in the head (start) of the list. All mentioned operations~-- insertion at the end, extraction (with a known pointer to a node), extraction from the head~-- have $\mathcal{O}(1)$ time complexity.

To embed \classname{FairLRU} in a container, an instance of \classname{FairLRU} should be added as a class member and container nodes should be derived from the \classname{FairLRU::Node} class (it contains data members that are required for a doubly linked list implementation). Then all basic operations, namely \findop, \insertop, and \removeop, should call corresponding methods on the \classname{FairLRU} instance.

\subsection{fair\_lfu.h and fair\_lfu.cpp}
\classname{FairLFU} external interface and usage scenario are similar to the ones in \classname{FairLRU}. However, it differs in the way it choses items for eviction (see \cref{sssec:lfu}).

Internally, the two-level linked list implementation is used, as described by Shah, Mitra, and Matani\cite{lfu}. This implementation has been chosen because it guarantees $\mathcal{O}(1)$ time complexity on all basic operations. Another well-known LFU implementation is based on a binary heap, but it achieves only $\mathcal{O}(\log N)$ time complexity.

The implementation used in the library differs from the original one. When a new item is inserted, the original implementation always assigns 1 to the node hit count. This yields a very ineffective behavior as the LFU tends to evict nodes that have just been added and preserves older ones, even those that have been accessed only twice.

The solution of this problem, presented in the library, is to calculate the initial hit counter value as a hit count of the least frequently accessed node (the one that is in the list head and may be evicted in the next step) incremented by one. This approach ensures that a new node is never inserted at the list head.

As a side effect, this adds the priority aging process (the approach is inverted~-- priority of new items is boosted instead of decreasing priority of older ones).

\subsection{weighted\_search\_tree.h}
Basing on \cite{park90} and \cite{park94}, the original weighted search tree has been reimplemented. Unlike the original implementation, the priority aging (see \cref{sssec:priority_aging}) is also implemented~-- while traversing over the AVL tree (during item lookup), priorities of all visited nodes are decreased, and the binary heap is adjusted accordingly.

The improved WST priority scheme (as discussed in \cref{sssec:improved_wst}) is used for the node priority representation. Another optimization is the elimination of an AVL balance factor as a separate structure member (in this case it takes at least 1 byte). It is embedded into the priority component~-- 2 bits are used for storing the balance factor, and remaining 30 are used for the priority (8 bits for the base priority and 22 bits for the accumulated priority). It is achieved with the \emph{C++ bit field} feature.

\subsection{hash\_table/}

\pathname{hash_table} folder contains \classname{FixedHashtableBase}, \classname{FixedHashtableBinaryHeap}, and \classname{FixedHashtableFairLU} classes.

\classname{FixedHashtableBase} defines hash table implementation, that is common for all derived classes. However, there is no logic for choosing a node to be evicted~-- \classname{FixedHashtableBase} forwards the call to its derived class, where this operation is implemented.

Normally, this polimorhic behavior can be achieved with a virtual method call. However, virtual invocation brings additional overhead. Another drawback is that a virtual call can not be inlined by a compiler. Note, that we are dealing with the static polymorphism~-- all functions that can be called are known at compile time and are never changed in runtime, in contrast to the dynamic polymorphism.

To simulate the static polymorphism, \emph{Curiously Recurring Template Pattern} is often used\cite{crtp}. The base class (that needs to call a function which implementation is provided only in the derived class) takes its derived class as a template argument. When a polymorphic method needs to be called, base class object casts \classname{this} pointer to the derived class and then calls the desired function. Name lookup mechanism finds the implementation of the method in the derived class and performs a call to it (see \cref{lst:crtp}). It is a regular call, so a compiler can apply call inlining and other optimizations.

\begin{listing}[tbp]
\caption{Curiously Recurring Template Pattern}
\label{lst:crtp}
\begin{minted}{cpp}
template <typename DerivedClass>
struct Base {
    void callFoo() {
        static_cast<DerivedClass*>(this)->foo();
    }
};
struct Derived : public Base<Derived> {
    void foo() {
        std::cout << "Derived foo" << std::endl;
    }
};

int main() {
    Derived d;
    d.callFoo(); //prints "Derived foo"
    return 0;
}
\end{minted}
\end{listing}

Two other classes, \classname{FixedHashtableFairLu} and \classname{FixedHashtableBinaryHeap}, derive \classname{FixedHashtableBase} and supplies the base class with the implementation of the method, that searches for the least valuable item, that is to be evicted. To choose the node, \classname{FixedHashtableFairLu} uses \classname{FairLRU} or \classname{FairLFU} manager (actually, the manager is passed as a template parameter, so it is possible to extend the implementation with a custom manager). \classname{FixedHashtableBinaryHeap} maintains a binary heap for item priorities. In fact, it acts very similar to the weighted search tree, but with the hash table used in place of the AVL tree.

\subsection{splay\_tree/}
\pathname{splay_tree} folder contains different variations of a splay tree. Practically the complete tree implementation is in \classname{SplayTreeBase} class. Similarly to the hash table implementation, the code for determining the least valuable item is excluded from the base class. Again, the derived classes are responsible for implementing it.

This splay tree implementation does not rely on parent pointer in any way. Therefore, it could be excluded from the node declaration. This would decrease the memory overhead per node. However, for some item eviction policies (LRU and LFU), a pointer to the parent is an inevitable requirement (for other policies it is not). To support both types of policies, a wrapper over a parent pointer is introduced. The wrapper interface consists of get and set operations. Two implementations of the wrapper are provided~-- an actual implementation, that encapsulates a real pointer, and a mock one, that stores no value.

Derived classes tell \classname{SplayTreeBase} (through a trait class) which wrapper implementation they require and \classname{SplayTreeBase} embeds the chosen wrapper into the \classname{SplayTreeBase::Node} structure.

Even though the mock wrapper contains no data members, the size of an empty structure can not be zero in C++\cite{sizeof_empty}. Therefore, when the wrapper is embedded into a node, it takes at least one byte while not containing any valuable information. This unnecessary overhead is eliminated using the \emph{Empty Member Optimization}\cite{ebo}.


There are two derived classes, that are responsible for the item eviction policy:
 \begin{itemize}[leftmargin=2cm]
 \item [\classname{SplayTreeFairLu}] is similar to the \classname{HashTableFairLu} class~-- it reuses LRU and LFU node eviction policies
 \item [\classname{SplayTreeBottomNode}] realizes the Splay eviction policy, as described in \cref{sssec:spolicy}
\end{itemize}

\subsection{Concurrent containers}
  \numdbname contains 2 concurrent adapters, namely \classname{CoarseLockConcurrentAdapter} and \classname{BinningConcurrentAdapter}, and \cndcname data structure implementation, described in \cref{sec:concurrent_containers}.



    \chapter{Performance evaluation}
    \label{ch:bench}
    This chapter describes the benchmarking program, its input parameters, and the specification of the machines, that were used for the performance evaluation. Additionally, the results of the benchmark runs are analyzed (the results itself are presented in Appendix B) .

\section{Benchmark}

The performance evaluations is done with the benchmarking program. For each tested data structure (presented in \Cref{ch:numdb} and \Cref{ch:alt}), the benchmark runs for a specified amount of time. Then the throughput (operations per second) is calculated as the total count of iterations divided by the total time.

Instead of separately measuring the performance of every basic operation (\findop, \insertop, \removeop), the overall numeric database performance is evaluated. The numeric database retrieval operation consists of either lookup (in case the item with the specified key is in the database) or lookup, user function invocation, item removal and item insertion. The user function is much slower than the database operations. Therefore, the most effective numerical database is the one that calls the user function as rarely as possible.

The recursive computation of the $N$th Fibonacci number is chosen as the user function. The algorithm implementation is trivia and, having the exponential time complexity, it is extremely inefficient. This is the advantage from the perspective of the benchmark as its task is to simulate a very computational-heavy function. What is more, the time it takes to compute the function can be easily adjusted by the function argument.

\pagebreak

Generally, cache systems perform well only on skewed inputs, when there is a small subset of items that are accessed most of the time while other items are accessed much less often. In this benchmark, a numerical database is tested with a random input sequence that has normal distribution of values. The parameters of the distribution are as follows:
\begin{description}
\item[Mean $u$] equals zero. The data structures presented in this library are agnostic to the particular argument values and their performance is only affected by the frequency of items in the input sequence. So $u$ can be any number. Zero is any number.

\item [Standard deviation $\sigma$] is derived from $Area{\mhyphen}under{\mhyphen}curve$ parameter and the available memory. $Capacity$ is calculated as the available memory divided by the size of a single item. $Area{\mhyphen}under{\mhyphen}curve$ is a value from the interval $(0,1)$. It defines the ratio of accesses to the $Capacity$ most valuable items to the total count of accesses. With given $Capacity$ and $Area{\mhyphen}under{\mhyphen}curve$, $\sigma$ is calculated as follows:
\begin{equation}
 \sigma = \frac{Capacity}{2 \times Quantile(0.5 + \frac{Area{\mhyphen}under{\mhyphen}curve}{2})}
 \end{equation}
\end{description}

\section{Benchmark Parameters}
The benchmark has several input parameters:
\begin{description}
\item [minval, maxval]-- the range of values the Fibonacci function is called with. It is defined as a range to simulate functions which execution time depends on its arguments.
\item [available memory]-- the maximum amount of memory the numerical database uses.
\item [thread count]-- relevant for concurrent numerical databases. Sets the number of  threads to run in parallel.
\item[mean changing rate]-- adjusts the rate at which the mean of the distribution is changed. If larger than zero, it simulates input sequences with non-static distribution. It is measured in the \emph{delta-per-iteration} units~-- its value is added to the mean at every iteration, e.g. if the rate is $^1/_{100}$, than after 1000 iteration the mean will move by 10.
\item[area-under-curve]-- the parameter that affects the standard deviation of the distribution.
\end{description}

\section{Analysis of the Sequential Containers}

\label{sec:secanalysis}

The performance evaluation of the sequential containers has been done on the following machine:

\begin{description}
    \item [CPU] Intel\textsuperscript{\textregistered{}} Core\textsuperscript{\texttrademark{}} i5-6200U
    2.30GHz (2.80 GHz\footnote{with Intel\textsuperscript{\textregistered{}} Turbo Boost Technology})$ \times$ 2 cores
    \item [Memory] 8 GB DDR3 1600 MHz
    \item [OS] Linux\textsuperscript{\textregistered{}} Ubuntu\textsuperscript{\textregistered{}} 16.04 LTS 64-bit
    \item [Compiler] GCC 5.4, compilation flags: \texttt{-O3 -std=c++14}
    \end{description}

In this chapter there are several graphs (\Cref{f1}, \Cref{f2}, \Cref{f3}), that visualises data presented in \Cref{ch:bresult}. The graphs do not include all the data from the tables, e.g. there is no graph for LRU and LFU based numerical databases; only the most representative candidates have been chosen. The outcomes have confirmed some assumptions, stated in the previous chapters, while refuted others.

The weighted search tree proved to be the most effective container for the numerical database. However, no concurrent WST implementation is known, so its usage is limited to the sequential environment.

The combination of the hash table and the binary heap performs about 10\% slower than WST. However, its advantage over WST is the simpler structure, that made it possible to develop the concurrent version of this container, called \cndcname.

The benchmark proved splay tree inefficiency. It is outperformed by WST and the hash table on all workloads. Therefore it is not recommended to use it for a numerical database.

\emph{Least-recently-used} and \emph{Least-frequently-used} eviction policies do not achieve the same performance as the policy, based on the binary heap. Possibly, because they do not rely on the initial item priority, thus give no preference to items that are rarely accessed  but took long to be calculated and due to that must be kept in the database.

\emph{Priority aging} does not show any advantage over the static priority. However, further studies and tests are required here, as the benchmark does not simulate the worst-case input for the static priority scheme.

\begin{figure}[h]
\centering
    \label{f1}
    \begin{tikzpicture}
      \begin{axis}[
        legend pos=outer north east,
        xlabel = Memory {[KiB]},
        ylabel = Throughput {[op/s]},
        xmin = 0.85, xmax = 3.15,
        xtick={1,2,3},
        xticklabels={4, 16, 64},
        ymin = 0, ymax = 2700,
        ytick={250, 500, 750, 1000, 1250, 1500, 1750, 2000, 2250, 2500},
        ymajorgrids=true,
        grid style=dashed,
      ]
        \addplot+[densely dashed, color=black, mark = no] coordinates{(1, 588)(2, 588)(3, 588)};
        \addplot coordinates{(1, 2015)(2, 2411)(3, 2329)};
        \addplot coordinates{(1, 2505)(2, 2258)(3, 2211)};
        \addplot+[color=blue] coordinates{(1, 879)(2, 887)(3, 896)};
        \legend{Baseline,WST,Hash table,Splay tree}
      \end{axis}
      \end{tikzpicture}
\caption{Sequential containers comparison (variable memory size)}
\end{figure}


\begin{figure}
\centering
    \label{f2}
    \begin{tikzpicture}
      \begin{axis}[
        legend pos=outer north east,
        xlabel = \emph{area-under-curve},
        ylabel = Throughput {[op/s]},
        xmin = 0.65, xmax = 6.35,
        xtick={1,2,3,4,5,6},
        xticklabels={0.45, 0.55, 0.65, 0.75, 0.85, 0.95},
        ymin = 0, ymax = 3000,
        ytick={250, 500, 750, 1000, 1250, 1500, 1750, 2000, 2250, 2500, 2750},
        ymajorgrids=true,
        grid style=dashed,
      ]
        \addplot+[densely dashed, color=black, mark = no] coordinates{(1, 588)(2, 588)(3, 588)(4, 588)(5,588)(6,588)};
        \addplot coordinates{(1, 1199)(2, 1432)(3, 1709)(4, 1997)(5,2329)(6,2890)};
        \addplot coordinates{(1, 1183)(2, 1416)(3, 1655)(4, 1904)(5,2211)(6,2733)};
        \addplot+[color=blue] coordinates{(1, 658)(2, 698)(3, 747)(4, 801)(5,896)(6,1040)};
        \legend{Baseline,WST,Hash table,Splay tree}
      \end{axis}
    \end{tikzpicture}
\caption{Sequential containers comparison (variable \emph{area-under-curve} value)}
\end{figure}


\begin{figure}[!t]
\centering
    \label{f3}
    \begin{tikzpicture}
          \begin{axis}[
            legend pos=outer north east,
            xlabel = Mean changing rate,
            ylabel = Throughput {[op/s]},
            xmin = 0.80, xmax = 4.20,
            xtick={1,2,3,4},
            xticklabels={0, $^1/{100}$, $^1/{10}$, 1},
            ymin = 0, ymax = 3000,
            ytick={250, 500, 750, 1000, 1250, 1500, 1750, 2000, 2250, 2500, 2750},
            ymajorgrids=true,
            grid style=dashed,
          ]
            \addplot+[densely dashed, color=black, mark = no] coordinates{(1, 588)(2, 588)(3, 588)(4, 588)};
            \addplot coordinates{(1, 2461)(2, 2329)(3, 1161)(4, 330)};
            \addplot coordinates{(1, 2344)(2, 2211)(3, 1238)(4, 336)};
            \addplot+[color=blue] coordinates{(1, 957)(2, 896)(3, 565)(4, 311)};
            \legend{Baseline,WST,Hash table,Splay tree}
          \end{axis}
        \end{tikzpicture}
\caption{Sequential containers comparison (variable mean changing rate)}
\end{figure}


\section{Analysis of the Concurrent Containers}

\label{sec:conanalysis}

The performance evaluation of the concurrent containers has been performed on the following machine:

\begin{description}
    \item [CPU] Intel\textsuperscript{\textregistered{}} Xeon\textsuperscript{\textregistered{}} E5-2650 v4
    2.20GHz (2.90 GHz\footnote{with Intel\textsuperscript{\textregistered{}} Turbo Boost Technology}) $ \times $ 2 sockets $ \times $ 12 cores
    \item [Memory] 256 GB DDR4 2400 MHz
    \item [OS] Linux\textsuperscript{\textregistered{}} Ubuntu\textsuperscript{\textregistered{}} 16.04 LTS 64-bit
    \item [Compiler] GCC 5.4, compilation flags: \texttt{-O3 -std=c++14}
    \end{description}

Same as with the sequential containers, the complete benchmark data is presented in \Cref{ch:bresult}. \Cref{f4} shows the performance difference between the tested concurrent containers. \Cref{f5} demonstrates their scalability~-- the relative per-thread performance with increasing number of threads. For example, the graph shows that with 24 threads running simultaneously, the performance of one thread in \cndcname is 30\% lower than that in the single-threaded test.

As expected, the coarse-grained locking approach yields the very ineffective container. The observation that with 2 threads the performance is two times lower compared to the single-threaded test (note, the total, not the per-thread performance) can be explained by the high contantion on the single mutex. The process of locking/unlocking an open mutex is quite fast operation, but locking the already locked mutex implies, that the thread will be suspended and then resumed. This is quite heavy and complex operation from the perspective of an operational system.

The binning approach performs better than the previous one. However, \Cref{f5} clearly shows, that it does not scale very good for the bigger number of threads~-- at 24 threads, per-thread performance is only 10\% as high as the maximum.

\cndcname shows excellent scalability, even with 24 threads. The poor per-thread performance is rather explained by the huge memory overhead per item than by computational complexity~-- in addition to the overhead, introduced by the data structure itself, synchronization adds at least 60 bytes for each item. The further studies should be directed at lowering this overhead. For example, some synchronization can be performed in the lock-free fashion with no mutexes involved.

\begin{figure}
\centering
    \label{f4}
    \begin{tikzpicture}
          \begin{axis}[
            legend pos=outer north east,
            xlabel = Thread count,
            ylabel = Throughput {[op/s]},
            ylabel shift = 0.5 pt,
            xmin = 0, xmax = 26,
            xtick={1, 2, 4, 8, 16, 24},
            ymin = 0, ymax = 1400,
            ytick={100, 200, 300, 400, 500, 600, 700, 800, 900, 1000, 1100, 1200, 1300},
            yticklabels={1000, 2000, 3000, 4000, 5000, 6000, 7000, 8000, 9000, 10000, 11000, 12000, 13000},
            ymajorgrids=true,
            grid style=dashed,
          ]
            \addplot coordinates{(1, 73.4)(2, 152.4)(4, 298.4)(8, 512.0)(16, 904.0)(24, 1292.0)};
            \addplot coordinates{(1, 234)(2, 123)(4, 120)(8, 127)(16, 124)(24, 127)};
            \addplot+[mark = triangle, mark options=solid]
                     coordinates{(1, 223)(2, 279)(4, 332)(8, 379)(16, 525)(24, 636)};

            \legend{\cndcname, Coarse + WST, Binning + WST}
          \end{axis}
        \end{tikzpicture}
\caption{Concurrent containers comparison (variable thread count)}
\end{figure}

\begin{figure}
\centering
    \label{f5}
    \begin{tikzpicture}
          \begin{axis}[
            legend pos=outer north east,
            xlabel = Thread count,
            ylabel = Relative performance per thread,
            xmin = 0, xmax = 26,
            xtick={1, 2, 4, 8, 16, 24},
            ymin = 0, ymax = 1.1,
            ytick={0.1, 0.2, 0.3, 0.4, 0.5, 0.6, 0.7, 0.8, 0.9, 1},
            ymajorgrids=true,
            grid style=dashed]
              \addplot coordinates{(1, 1)(2, 1.03)(4, 0.97)(8, 0.87)(16, 0.77)(24, 0.70)};
              \addplot coordinates{(1, 1)(2, 0.26)(4, 0.127)(8, 0.067)(16, 0.033)(24, 0.02)};
              \addplot+[mark = triangle, mark options=solid]
                       coordinates{(1, 1)(2, 0.626)(4, 0.37)(8, 0.212)(16, 0.147)(24, 0.119)};
              \legend{\cndcname, Coarse + WST, Binning + WST}
          \end{axis}
    \end{tikzpicture}
    \caption{Concurrent containers scalability comparison}
\end{figure}


\setsecnumdepth{part}
    \chapter{Conclusion}
    \label{ch:conclusion}
    \section{Bla-bla}

bla bla


\bibliographystyle{iso690}
\bibliography{bibliography}

\setsecnumdepth{all}
\appendix

\chapter{Benchmark results}
\section{Container reference}
\begin{description}
\item[Baseline] The performance of the user function itself without any numerical database.
\item[WST, aging = 0] The original weighted search tree, as described in \cite{park94}.
\item[WST, aging = $N$] Weighted search tree with aging mechanism. For every traversed node, its priority is decreased by $N \times 256$
\item[Hash table, Binary heap, aging = $N$] The container, very similar to the WST, but with a hash table in place of a BST. $N$ parameter has the same meaning as for the WST; 0 value means that no priority aging is performed.
\item[Hash table, LRU] The hash table, that uses LRU strategy for item eviction.
\item[Hash table, LFU] The hash table, that uses LFU strategy for item eviction.

\item[Splay tree, Splay policy, canonical] The splay tree with the splay policy (as described in \cref{sssec:spolicy}). ``Canonical'' stands for the splaying strategy~-- the node is always splayed up to the root\cite{splay_tree}.
\item[Splay tree, Splay policy, partial] The splay tree with the splay policy (as described in \cref{sssec:spolicy}). ``Partial'' stands for the splaying strategy~-- each node has the access counter and it is splayed until its counter value is less than that of its parent the node. This approach is known as the \emph{partial splaying}\cite{partial_splaying}.
\item[Splay tree, LRU, canonical] The splay tree with the LRU policy.
\item[Splay tree, LFU, canonical] The splay tree with the LFU policy.
\end{description}
\pagebreak

\begin{tabular}[h]{l r} \toprule
Parameter & Value \\ \midrule
Min/Max arg & 25/35 \\
Memory & \emph{vary} \\
Area-under-curve & 0.85 \\
Mean changing rate & 0 \\ \bottomrule
\end{tabular}

\begin{table}
\caption{Variable memory size (static input distribution)}
\begin{tabular}[b]{l  r  r  r } \toprule
Container & \multicolumn{3}{c}{Throughput [op/s]} \\ \cmidrule(r){2-4}
& 4 KiB & 16 KiB & 64 KiB \\ \midrule
Baseline & \multicolumn{3}{r}{588}  \\
\\
WST, aging = 0 & 2265 & 2595 & 2461 \\
WST, aging = 1 & 2502 & 2568 & 2279 \\
WST, aging = 2 & 2619 & 2369 & 2052 \\
WST, aging = 4 & 2651 & 2040 & 1699 \\
WST, aging = 16 & 1500 & 1741 & 1721 \\
\\
Hash table, Binary heap, aging = 0 & 2716 & 2407 & 2344 \\
Hash table, Binary heap, aging = 1 & 2686 & 2388 & 2353 \\
Hash table, Binary heap, aging = 2 & 2640 & 2422 & 2391 \\
Hash table, Binary heap, aging = 4 & 2547 & 2456 & 2335 \\
Hash table, Binary heap, aging = 16 & 2568 & 2344 & 2316 \\
Hash table, LRU & 803 & 779 & 771 \\
Hash table, LFU & 750 & 711 & 706 \\
\\
Splay tree, Splay policy, canonical & 1014 & 976 & 957 \\
Splay tree, Splay policy, partial & 766 & 963 & 723 \\
Splay tree, LRU, canonical & 721 & 700 & 679 \\
Splay tree, LFU, canonical & 671 & 663 & 659 \\
\bottomrule
\end{tabular}
\end{table}


%mem, rate = 1
\pagebreak

\begin{tabular}[h]{l r} \toprule
Parameter & Value \\ \midrule
Min/Max arg & 25/35 \\
Memory & \emph{vary} \\
Area-under-curve & 0.85 \\
Mean changing rate & $^1/_{100}$ \\ \bottomrule
\end{tabular}

\begin{table}
\caption{Variable memory size (time-varying input distribution)}
\begin{tabular}[]{l  r  r  r } \toprule
Container & \multicolumn{3}{c}{Throughput [op/s]} \\ \cmidrule(r){2-4}
& 4 KiB & 16 KiB & 64 KiB \\ \midrule
Baseline & \multicolumn{3}{r}{588}  \\
\\
WST, aging = 0 & 2015 & 2411 & 2329 \\
WST, aging = 1 & 2175 & 2348 & 2183 \\
WST, aging = 2 & 2368 & 2265 & 1931 \\
WST, aging = 4 & 2430 & 1864 & 1609 \\
WST, aging = 16 & 1343 & 1514 & 1560 \\
\\
Hash table, Binary heap, aging = 0 & 2505 & 2258 & 2211 \\
Hash table, Binary heap, aging = 1 & 2320 & 2245 & 2220 \\
Hash table, Binary heap, aging = 2 & 2295 & 2176 & 2199 \\
Hash table, Binary heap, aging = 4 & 2354 & 2202 & 2204 \\
Hash table, Binary heap, aging = 16 & 2121 & 2167 & 2152 \\
Hash table, LRU & 697 & 711 & 720 \\
Hash table, LFU & 639 & 646 & 651 \\
\\
Splay tree, Splay policy, canonical & 879 & 887 & 896 \\
Splay tree, Splay policy, partial & 812 & 872 & 799 \\
Splay tree, LRU, canonical & 640 & 634 & 631 \\
Splay tree, LFU, canonical & 610 & 626 & 628 \\
\bottomrule
\end{tabular}
\end{table}

%area
\pagebreak

\begin{tabular}[h]{l r} \toprule
Parameter & Value \\ \midrule
Min/Max arg & 25/35 \\
Memory & 64 KiB \\
Area-under-curve & \emph{vary} \\
Mean changing rate & $^1/_{100}$ \\ \bottomrule
\end{tabular}

\begin{table}
\caption{Variable \emph{area-under-curve} value}
\begin{tabular}[]{l r r r r r r} \toprule
Container & \multicolumn{6}{c}{Throughput [op/s]} \\ \cmidrule(r){2-7}
& 0.45 & 0.55 & 0.65 & 0.75 & 0.85 & 0.95 \\ \midrule
Baseline & \multicolumn{6}{r}{588}  \\
\\
WST, aging = 0 & 1199 & 1432 & 1709 & 1997 & 2329 & 2890 \\
WST, aging = 1 & 1103 & 1341 & 1541 & 1802 & 2183 & 2749 \\
WST, aging = 2 & 987 & 1188 & 1375 & 1630 & 1931 & 2559 \\
WST, aging = 4 & 858 & 983 & 1146 & 1321 & 1609 & 2325 \\
WST, aging = 16 & 798 & 887 & 1028 & 1216 & 1560 & 2537 \\
\\
Hash table, Binary heap, aging = 0 & 1183 & 1416 & 1655 & 1904 & 2211 & 2733 \\
Hash table, Binary heap, aging = 1 & 1185 & 1405 & 1651 & 1881 & 2220 & 2759 \\
Hash table, Binary heap, aging = 2 & 1178 & 1403 & 1647 & 1897 & 2199 & 2784 \\
Hash table, Binary heap, aging = 4 & 1188 & 1407 & 1651 & 1908 & 2204 & 2755 \\
Hash table, Binary heap, aging = 16 & 1179 & 1391 & 1638 & 1845 & 2152 & 2675 \\
Hash table, LRU & 603 & 621 & 643 & 678 & 720 & 809 \\
Hash table, LFU & 576 & 591 & 604 & 624 & 651 & 700 \\
\\
Splay tree, Splay policy, canonical & 658 & 698 & 747 & 801 & 896 & 1040 \\
Splay tree, Splay policy, partial & 616 & 669 & 767 & 781 & 799 & 861 \\
Splay tree, LRU, canonical & 550 & 566 & 582 & 595 & 631 & 687 \\
Splay tree, LFU, canonical & 567 & 577 & 590 & 608 & 628 & 675 \\
\bottomrule
\end{tabular}
\end{table}

%momentum
\pagebreak

\begin{tabular}[h]{l r} \toprule
Parameter & Value \\ \midrule
Min/Max arg & 25/35 \\
Memory & 64 KiB \\
Area-under-curve & 0.85 \\
Mean changing rate & \emph{vary} \\ \bottomrule
\end{tabular}

\begin{table}
\caption{Variable mean changing rate}
\begin{tabular}[]{l r r r r r} \toprule
Container & \multicolumn{4}{c}{Throughput [op/s]} \\ \cmidrule(r){2-5}
& 0 & $^1/_{100}$ & $^1/_{10}$ & 1 \\ \midrule
Baseline & \multicolumn{4}{r}{588}  \\
\\
WST, aging = 0 & 2461 & 2329 & 1161 & 330 \\
WST, aging = 1 & 2279 & 2183 & 1169 & 344 \\
WST, aging = 2 & 2052 & 1931 & 1136 & 349 \\
WST, aging = 4 & 1699 & 1609 & 1054 & 350 \\
WST, aging = 16 & 1721 & 1560 & 800 & 334 \\
\\
Hash table, Binary heap, aging = 0 & 2344 & 2211 & 1238 & 336 \\
Hash table, Binary heap, aging = 1 & 2353 & 2220 & 1238 & 333 \\
Hash table, Binary heap, aging = 2 & 2391 & 2199 & 1241 & 331 \\
Hash table, Binary heap, aging = 4 & 2335 & 2204 & 1235 & 331 \\
Hash table, Binary heap, aging = 16 & 2316 & 2152 & 1214 & 333 \\
Hash table, LRU & 771 & 720 & 479 & 386 \\
Hash table, LFU & 706 & 651 & 438 & 356 \\
\\
Splay tree, Splay policy, canonical & 957 & 896 & 565 & 311 \\
Splay tree, Splay policy, partial & 723 & 799 & 533 & 398 \\
Splay tree, LRU, canonical & 679 & 631 & 419 & 350 \\
Splay tree, LFU, canonical & 659 & 628 & 418 & 340 \\
\bottomrule
\end{tabular}
\end{table}


\chapter{Acronyms}

\begin{acronym}
	\acro{BST}{Binary Search Tree}
	\acro{$T$}{A binary tree}
	\acro{$N$}{A node in a binary tree or a hash table}
	\acro{$K$}{A key, used for the lookup in a container}
	\acro{$F$}{User-provided function that accepts $K$ as an argument}
	\acro{$R$}{A result, that is calculated from $K$ by $F$}
\end{acronym}


\chapter{Contents of enclosed CD}
\dirtree{%
    .1 /.
    .2 Code\DTcomment{\numdbname project folder}.
      .3 3rdparty\DTcomment{\numdbname additional libraries}.
        .4 function\_traits.
        .4 google\_benchmark.
        .4 gtest.
        .4 murmurhash2functor.
      .3 benchmark\DTcomment{\numdbname benchmark source files}.
      .3 include\DTcomment{\numdbname library header files}.
        .4 numdb.
          .5 cndc.
          .5 concurrent\_adapters.
          .5 hash\_table.
          .5 splay\_tree.
          .5 wst.
      .3 lib\DTcomment{\numdbname library source code files}.
      .3 test\DTcomment{\numdbname unit tests}.
    .2 Text\DTcomment{this thesis \LaTeX{} source code folder}.
      .3 pdf.
        .4 thesis{.}pdf\DTcomment{this thesis in the PDF format}.
      .3 tex\DTcomment{\LaTeX{} source code}.
    }
\end{document}
