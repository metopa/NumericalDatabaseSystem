\section{Benchmark}

Performance evaluations is done with the benchmark program. For each tested data structure, the benchmark runs for a specified amount of time. Then the throughput \(operations per second\) is calculated as the total count of iterations divided by the total time.

Instead of separately measuring the performance of every basic operation \(look-up, insertion, removal\), an overall numeric database performance is evaluated. The numeric database retrieval operations consists of either look-up \(in case the item with the specified key is in the database\) or look-up, user function invocation, item removal and item insertion. The user function is much slower than database operations. Therefore practically the most optimal numerical database is the one that calls a user function as rarely as possible.

As a user function the recursive computation of the Nth FIbonacci number is chosen. This algorithm has a trivial implementation and having an exponential time complexity it is extremely inefficient. It is an advantage from the perspective of the benchmark as its task is to simulate a very computational-heavy function. What is more, the time it takes to compute the function can be easily adjusted by the function argument.

Generally, cache systems perform well only on skewed inputs, when there is a small subset of items that are accessed most of the time while other items are accessed much less often. In this benchmark, a numerical database is tested with a random input sequence that has normal distribution of values. The parameters of the distribution are as follows:
Mean [u]  is zero. The data structures presented in this library are agnostic to the particular argument values and their performance is only affected by the frequency of items in the input sequence. So [u] can equal to any number. Zero is any number.
Standard deviation [sigma] is computed from the available memory and area-under-curve parameter. Capacity is calculated as the available memory divided by the size of a single item. Area-under-curve is a value from the interval \(0,1\) exclusive. It defines the ratio of accesses to the N most valuable items to the total count of accesses, where N is capacity. With given capacity and area-under-curve, [sigma] equals to capacity / 2 / Quantile(N(0,1), 0.5 + area / 2).

The benchmark has several input parameters:
User function argument minimum and maximum values – the range of values the Fibonacci function is called with is defined to simulate function which speed depends on its arguments.
Available memory – the maximum amount of memory a numerical database uses.
Thread count – relevant for concurrent numerical databases. Sets the number of  threads to run in parallel.
Mean momentum – adjusts the rate at which the mean of the distribution is changed. If larger than zero, it simulates input sequences with non-static distribution.
Area under curve – the parameter that affects the standard deviation of the distribution.

The performance evaluation is done on those machines:
Machine A
%CPU: Intel® Core™ i5-6200U CPU @ 2.30GHz × 2
Memory: 8 GB DDR3 1600 MHz
OS: Linux Ubuntu 16.04 LTS 64-bit
Machine B
CPU:
Memory:
OS:

\section{Results}
