In this chapter the contributions of~\cite{park90}~-- the numerical database system and the wighted search tree~-- are explained in~detail.

\section{Concept}
In the year 1990 S.C.~Park, J.P.~Draayer, and S.-Q.~Zheng introduced a memoization system, called \emph{numerical database}\cite{park90}. Like every memoization system, its primary goal is to reduce costly redundant calculation. The main idea behind this concept is to design a data structure that stores a limited number of items~-- key-value pairs~-- and provide an efficient way to retrieve, insert and remove items. A value is associated with the key that can be used to calculate the value by a function. The complete process of the item retrieval is demonstrated in \Cref{alg:numdb_get}.

%\newfloat{algorithm}{tp}{lop}
\begin{algorithm}[t]
\caption{Numerical database item retrieval}\label{alg:numdb_get}
\begin{algorithmic}[1]
  \Procedure{Retrieve}{$numdb, K$}
    \State $(V, found) \gets numdb.\func{find}{K} $
    \If {$found = true$}
      \State $numdb.\func{update\_priority}{K} $ \Comment{Can be embedded into \findop}
      \State \textbf{return} $V$
    \EndIf
    \State \Comment{The item was not found and must be recalculated}
    \State $(V, priority) \gets numdb.\func{call\_user\_function}{K}$

    \If {$numdb.\func{is\_full}{}$}
      \State $numdb.\func{evict\_item}{}$ \Comment{Remove the item with the lowest priority}
    \EndIf
    \State $numdb.\func{insert}{K, V, priority}$
    \State \textbf{return} $V$
  \EndProcedure
\end{algorithmic}
\end{algorithm}

\section{Priority}
\label{sec:wst_priority}

Each item has its assigned priority. When accessed, the priority of the node is updated. Park et. al. define priority as follows~-- ``A good priority strategy should enable the frequency of a data item and its intrinsic value to be incorporated into its assigned priority.'' Initial priority can be supplied by an external algorithm or computed using some heuristics. For example, the heuristic can be based on the time it took to compute the value: some items can take much more time to be calculated than others, then these elements should be kept in the database even if they are accessed relatively rarely.

When a numerical database reaches its maximum capacity, the item with the lowest priority is removed from the database prior to the next insertion. This implies that the database should support queries on the item with the minimum priority. The way it is achieved in the original proposal is discussed in \Cref{sec:wst}.
\\

Park et. al. introduced a space-optimal representation of the item priority. To distinguish this representation from others, it will be called the \emph{weighted search tree priority} (WST priority). This representation is stored in a single 32 bits long unsigned integer, but combines both the base priority and the hit frequency at the same time. The first 8 bits of the number are reserved for the base priority while remaining 24 bits contains hit frequency, multiplied by a base priority.
Therefore, the actual priority value equals to
\begin{equation}
hit\_frequency \times base\_priority \times 256 + base\_priority
\end{equation}

Another advantage of the WST priority is a simple adjustment procedure when the hit frequency is updated (see \Cref{alg:wst_priority1}). However, this representation also has some drawbacks that will be discussed in \Cref{sssec:improved_wst}.
%\newfloat{algorithm}{tbp}{lop}
\begin{algorithm}[t]
\caption{$WST$ priority update}\label{alg:wst_priority1}
\begin{algorithmic}[1]
  \Procedure{Update}{$priority$}\Comment{4 bytes long unsigned integer}
    \State $base\_priority\gets priority \mathrel{\&} \nsnum{FF}{16}$\Comment{Hexadecimal number}
    \State $priority \gets priority + base\_priority \times \nsnum{100}{16}$
    \State \Comment{Multiplication shifts the number 8 bits to the left}
    \State \textbf{return} $priority$
  \EndProcedure
\end{algorithmic}
\end{algorithm}

\section{Weighted Search Tree}
\label{sec:wst}
Park et. al. also proposed a data structure, called weighted search tree (\emph{WST}), that can be used as a base for the numerical database. The weighted search tree is a combination of two well-known data structures~-- the AVL tree and the binary heap. Each of them fulfills its purpose:
\begin{block-description}
\blockitem [AVL tree] is used for a fast item lookup, insertion, and removal in $\mathcal{O}(\log{\func{size}{T}})$ time.
\blockitem [Binary heap] is used to maintain priorities of nodes. Specifically, it provides an ability to find a node with the lowest priority in $\mathcal{O}(1)$ and can perform insertions and deletions in $\mathcal{O}(\log{\func{size}{T}})$ time.
\end{block-description}

Weighted search tree holds all its nodes in a single contiguous array. These nodes are ordered in the same way as in a regular minimal binary heap:
\begin{equation}
\begin{aligned}
  \forall N \in \textsc{Heap},{}
     & \func{priority}{N} < \func{priority}{\func{heapLeft}{N}} {} \land \\
     & \func{priority}{N} < \func{priority}{\func{heapRight}{N}}
\end{aligned}
\end{equation}

\func{heapLeft}{N} and \func{heapRight}{N} of $N$ with index $i$ are defined as $(2 \times i)$ and $(2 \times i + 1)$ respectively.


The difference is that each node is an AVL tree node at the same time~-- it stores links to its left child, right child, and parent node.

With a given structure the three basic operations are defined as follows:
\begin{block-description}
\blockitem[\findop] is the same as in any binary tree~-- it is begins at the root and then continues as described in \Cref{alg:bst_find}.
\blockitem[\insertop] consists of two steps:
  \begin{enumerate}
  \item The node is inserted into the binary search tree, and the tree is adjusted using AVL rotations. Obviously, these rotations only change pointers inside nodes. Therefore, they do not affect the binary heap structure.
  \item The node is inserted into the binary heap. It is done using the \func{heapify}{} operation\cite{sedgewick}. Note that heap adjustments would \emph{reorder nodes} so that binary search tree pointers would point to wrong nodes and the whole \emph{tree would become ill-formed}. To avoid this, special care should be taken when swapping nodes~-- one should also check and adjust tree pointers during swaps.
  \end{enumerate}
\blockitem[\removeop] is similar to the insertion, but steps are performed in the reverse order~-- at first, the element is removed from the heap, then from the tree. Again, heap adjustments should be performed with respect to the tree structure.
\end{block-description}

\section{Known Implementations}

There are several known implementations of the numerical database concept.

The reference implementation was made by Park, Bahri, Draayer, and Zheng. A complete source code is available at \cite{wstree}. It was implemented in Fortran programming language. However, the source code cannot be compiled with any modern Fortran compiler\cite{masat}. Therefore, this implementation is mentioned here but cannot participate in the benchmark. The \numdbname library contains the implementation of the weighted search tree, recreated basing on \cite{park90} and \cite{park94}.

Another implementation has been developed and described by Miroslav Masat in his bachelor’s thesis \cite{masat}. Sources are available at \cite{ccherish}. However in this particular implementation item priorities are defined solely by the user and are not updated afterward. Therefore it is only applicable in scenarios when the item importance is known beforehand. A canonical numerical database has a broader field of applications.

Both of the implementations are designed for a single-threaded environment only. It is a huge drawback since it is natural to run scientific applications on many-core systems. Therefore, a memoization system must support concurrent access in order to be usable in real-world applications. It was achieved in the \numdbname library.
